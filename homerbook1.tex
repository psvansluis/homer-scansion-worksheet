% !TeX root = homerbook1.tex
\input tikz
\special{papersize=105mm,297mm}
\hsize=88mm
\hoffset=-18mm
\voffset=-5mm
\vsize=255mm
\baselineskip=5.75ex
\def\nstanza{\vskip 5.75ex plus 1.75ex minus 1.75ex}
\font\tenrm="Fira Code" at 7pt
\font\mar="Fira Code:+zero;+onum" at 6pt
\font\tentt="Fira Code" at 7pt
\font\big="Fira Code" at 10pt
\font\bigish="Fira Code" at 8pt
\def\lineno#1{\strut\vadjust{\kern-\dp\strutbox\smash{{\mar #1 }}\kern\dp\strutbox}}
\catcode`_ = 13 % For macron
\catcode`ỽ = 13 % For breve
\catcode`× = 13 % For anceps
\catcode`| = 13 % For feet
\catcode`æ = 13 % For cæsura
\colorlet{mygreen}{green!60!gray}
\colorlet{myblue}{blue!60!gray}
\colorlet{myred}{red!60!gray}
\def_{%
    \tikzpicture[line cap=round,mygreen,overlay, remember picture]%
    \draw[-](0.1em,2.5ex) -- (0.6em,2.5ex);%
    \endtikzpicture%
}
\defỽ{%
    \tikzpicture[line cap=round,mygreen,overlay, remember picture]%
    \draw[-](0.02em,2.85ex) to[bend right=90] (0.5em,2.8ex);%
    \endtikzpicture%
}
\def×{%
    \tikzpicture[line cap=round,mygreen,overlay, remember picture]%
    \draw[-](0.2em,2.5ex) -- (0.8em,3.6ex);[overlay, remember picture]%
    \draw[-](0.2em,3.6ex) -- (0.8em,2.5ex);%
    \endtikzpicture%
}
\def|{%
    \tikzpicture[line cap=round,myblue,overlay, remember picture]%
    \draw[-](0,1.5ex) to (0em,3.5ex);%
    \endtikzpicture%
}
\defæ{%
    \tikzpicture[line cap=round,myred,overlay, remember picture]%
    \draw[-, rounded corners=.1em](0,-1ex) -- (0.2em,-1ex) -- (0.2em,3ex) -- (0,3ex);%
    \draw[-, rounded corners=.1em](.6em,-1ex) -- (0.4em,-1ex) -- (0.4em,3ex) -- (.6em,3ex);%
    \endtikzpicture%
}
\tt

\centerline{\big Homer’s Odyssey book 1}
\centerline{\bigish A worksheet for scansion}
\bgroup\obeylines

_Ἄνδρỽα μỽοι |_ἔννỽεπỽε, |Μ_οῦσỽα,æ πỽο|λ_ύτρỽοπỽον, |_ὃς μỽάλỽα |π_ολ×λὰ
πλ_άγχθỽη, ỽἐ|π_εὶ Τρ_οί|_ηςæ ỽἱỽε|ρ_ὸν πτỽολỽί|_εθρỽον ỽἔ|π_ερσ×εν·
π_ολλ_ῶν |δ’ _ἀνθ_ρώπ|_ωνæ ỽἴδỽεν |_ἄστỽεỽα |κ_αὶ νỽόỽον |_ἔγ×νω,
π_ολλỽὰ δ’ ỽὅ |γ’ _ἐν π_όν|_τῳæ πỽάθỽεν |_ἄλγỽεỽα |_ὃν κỽατỽὰ |θ_υμ×όν,
_ἀρνỽύμỽε|ν_ος _ἥν |_τεæ _ψυ|χ_ὴν κ_αὶ |ν_όστỽον ỽἑ|τ_αίρ×ων.
_ἀλλ' _οὐδ' |_ὧς ỽἑτỽά|ρ_ουςæ _ἐρ|ρ_ύσỽατỽο, |_ἱỽέμỽεν|_ός π×ερ·
αὐτῶν γὰρ σφετέρῃσιν ἀτασθαλίῃσιν ὄλοντο,
νήπιοι, οἳ κατὰ βοῦς Ὑπερίονος Ἠελίοιο
ἤσθιον· αὐτὰρ ὁ τοῖσιν ἀφείλετο νόστιμον ἦμαρ.
τῶν ἁμόθεν γε, θεά, θύγατερ Διός, εἰπὲ καὶ ἡμῖν.    \lineno{10}
\nstanza
ἔνθ' ἄλλοι μὲν πάντες, ὅσοι φύγον αἰπὺν ὄλεθρον,
οἴκοι ἔσαν, πόλεμόν τε πεφευγότες ἠδὲ θάλασσαν·
τὸν δ' οἶον, νόστου κεχρημένον ἠδὲ γυναικός,
νύμφη πότνι' ἔρυκε Καλυψώ, δῖα θεάων,
ἐν σπέεσι γλαφυροῖσι, λιλαιομένη πόσιν εἶναι.
ἀλλ' ὅτε δὴ ἔτος ἦλθε περιπλομένων ἐνιαυτῶν,
τῷ οἱ ἐπεκλώσαντο θεοὶ οἶκόνδε νέεσθαι
εἰς Ἰθάκην, οὐδ' ἔνθα πεφυγμένος ἦεν ἀέθλων
καὶ μετὰ οἷσι φίλοισι· θεοὶ δ' ἐλέαιρον ἅπαντες
νόσφι Ποσειδάωνος· ὁ δ' ἀσπερχὲς μενέαινεν    \lineno{20}
ἀντιθέῳ Ὀδυσῆϊ πάρος ἣν γαῖαν ἱκέσθαι.
ἀλλ' ὁ μὲν Αἰθίοπας μετεκίαθε τηλόθ' ἐόντας,
Αἰθίοπας, τοὶ διχθὰ δεδαίαται, ἔσχατοι ἀνδρῶν,
οἱ μὲν δυσομένου Ὑπερίονος, οἱ δ' ἀνιόντος,
ἀντιόων ταύρων τε καὶ ἀρνειῶν ἑκατόμβης.
ἔνθ' ὅ γε τέρπετο δαιτὶ παρήμενος· οἱ δὲ δὴ ἄλλοι
Ζηνὸς ἐνὶ μεγάροισιν Ὀλυμπίου ἁθρόοι ἦσαν.
τοῖσι δὲ μύθων ἦρχε πατὴρ ἀνδρῶν τε θεῶν τε·
μνήσατο γὰρ κατὰ θυμὸν ἀμύμονος Αἰγίσθοιο,
τόν ῥ' Ἀγαμεμνονίδης τηλεκλυτὸς ἔκταν' Ὀρέστης·    \lineno{30}
τοῦ ὅ γ' ἐπιμνησθεὶς ἔπε' ἀθανάτοισι μετηύδα·
\nstanza
«ὢ πόποι, οἷον δή νυ θεοὺς βροτοὶ αἰτιόωνται.
ἐξ ἡμέων γάρ φασι κάκ' ἔμμεναι· οἱ δὲ καὶ αὐτοὶ
σφῇσιν ἀτασθαλίῃσιν ὑπὲρ μόρον ἄλγε' ἔχουσιν,
ὡς καὶ νῦν Αἴγισθος ὑπὲρ μόρον Ἀτρεΐδαο
γῆμ' ἄλοχον μνηστήν, τὸν δ' ἔκτανε νοστήσαντα,
εἰδὼς αἰπὺν ὄλεθρον, ἐπεὶ πρό οἱ εἴπομεν ἡμεῖς,
Ἑρμείαν πέμψαντες, ἐΰσκοπον Ἀργεϊφόντην,
μήτ' αὐτὸν κτείνειν μήτε μνάασθαι ἄκοιτιν·
ἐκ γὰρ Ὀρέσταο τίσις ἔσσεται Ἀτρεΐδαο,    \lineno{40}
ὁππότ' ἂν ἡβήσῃ τε καὶ ἧς ἱμείρεται αἴης.
ὣς ἔφαθ' Ἑρμείας, ἀλλ' οὐ φρένας Αἰγίσθοιο
πεῖθ' ἀγαθὰ φρονέων· νῦν δ' ἁθρόα πάντ' ἀπέτεισε.»
\nstanza
τὸν δ' ἠμείβετ' ἔπειτα θεὰ γλαυκῶπις Ἀθήνη·
«ὦ πάτερ ἡμέτερε Κρονίδη, ὕπατε κρειόντων,
καὶ λίην κεῖνός γε ἐοικότι κεῖται ὀλέθρῳ,
ὡς ἀπόλοιτο καὶ ἄλλος ὅτις τοιαῦτά γε ῥέζοι.
ἀλλά μοι ἀμφ' Ὀδυσῆϊ δαΐφρονι δαίεται ἦτορ,
δυσμόρῳ, ὃς δὴ δηθὰ φίλων ἄπο πήματα πάσχει
νήσῳ ἐν ἀμφιρύτῃ, ὅθι τ' ὀμφαλός ἐστι θαλάσσης,    \lineno{50}
νῆσος δενδρήεσσα, θεὰ δ' ἐν δώματα ναίει,
Ἄτλαντος θυγάτηρ ὀλοόφρονος, ὅς τε θαλάσσης
πάσης βένθεα οἶδεν, ἔχει δέ τε κίονας αὐτὸς
μακράς, αἳ γαῖάν τε καὶ οὐρανὸν ἀμφὶς ἔχουσι.
τοῦ θυγάτηρ δύστηνον ὀδυρόμενον κατερύκει,
αἰεὶ δὲ μαλακοῖσι καὶ αἱμυλίοισι λόγοισι
θέλγει, ὅπως Ἰθάκης ἐπιλήσεται· αὐτὰρ Ὀδυσσεύς,
ἱέμενος καὶ καπνὸν ἀποθρῴσκοντα νοῆσαι
ἧς γαίης, θανέειν ἱμείρεται. οὐδέ νυ σοί περ
ἐντρέπεται φίλον ἦτορ, Ὀλύμπιε; οὔ νύ τ' Ὀδυσσεὺς    \lineno{60}
Ἀργείων παρὰ νηυσὶ χαρίζετο ἱερὰ ῥέζων
Τροίῃ ἐν εὐρείῃ; τί νύ οἱ τόσον ὠδύσαο, Ζεῦ;»
\nstanza
τὴν δ' ἀπαμειβόμενος προσέφη νεφεληγερέτα Ζεύς·
«τέκνον ἐμόν, ποῖόν σε ἔπος φύγεν ἕρκος ὀδόντων.
πῶς ἂν ἔπειτ' Ὀδυσῆος ἐγὼ θείοιο λαθοίμην,
ὃς περὶ μὲν νόον ἐστὶ βροτῶν, περὶ δ' ἱρὰ θεοῖσιν
ἀθανάτοισιν ἔδωκε, τοὶ οὐρανὸν εὐρὺν ἔχουσιν;
ἀλλὰ Ποσειδάων γαιήοχος ἀσκελὲς αἰὲν
Κύκλωπος κεχόλωται, ὃν ὀφθαλμοῦ ἀλάωσεν,
ἀντίθεον Πολύφημον, ὅου κράτος ἐστὶ μέγιστον    \lineno{70}
πᾶσιν Κυκλώπεσσι· Θόωσα δέ μιν τέκε νύμφη,
Φόρκυνος θυγάτηρ, ἁλὸς ἀτρυγέτοιο μέδοντος,
ἐν σπέεσι γλαφυροῖσι Ποσειδάωνι μιγεῖσα.
ἐκ τοῦ δὴ Ὀδυσῆα Ποσειδάων ἐνοσίχθων
οὔ τι κατακτείνει, πλάζει δ' ἀπὸ πατρίδος αἴης.
ἀλλ' ἄγεθ' ἡμεῖς οἵδε περιφραζώμεθα πάντες
νόστον, ὅπως ἔλθῃσι· Ποσειδάων δὲ μεθήσει
ὃν χόλον· οὐ μὲν γάρ τι δυνήσεται ἀντία πάντων
ἀθανάτων ἀέκητι θεῶν ἐριδαινέμεν οἶος.»
\nstanza
τὸν δ' ἠμείβετ' ἔπειτα θεὰ γλαυκῶπις Ἀθήνη·    \lineno{80}
«ὦ πάτερ ἡμέτερε Κρονίδη, ὕπατε κρειόντων,
εἰ μὲν δὴ νῦν τοῦτο φίλον μακάρεσσι θεοῖσι,
νοστῆσαι Ὀδυσῆα πολύφρονα ὅνδε δόμονδε,
Ἑρμείαν μὲν ἔπειτα, διάκτορον Ἀργεϊφόντην,
νῆσον ἐς Ὠγυγίην ὀτρύνομεν, ὄφρα τάχιστα
νύμφῃ ἐϋπλοκάμῳ εἴπῃ νημερτέα βουλήν,
νόστον Ὀδυσσῆος ταλασίφρονος, ὥς κε νέηται.
αὐτὰρ ἐγὼν Ἰθάκηνδε ἐλεύσομαι, ὄφρα οἱ υἱὸν
μᾶλλον ἐποτρύνω καί οἱ μένος ἐν φρεσὶ θείω,
εἰς ἀγορὴν καλέσαντα κάρη κομόωντας Ἀχαιοὺς    \lineno{90}
πᾶσι μνηστήρεσσιν ἀπειπέμεν, οἵ τέ οἱ αἰεὶ
μῆλ' ἁδινὰ σφάζουσι καὶ εἰλίποδας ἕλικας βοῦς.
πέμψω δ' ἐς Σπάρτην τε καὶ ἐς Πύλον ἠμαθόεντα
νόστον πευσόμενον πατρὸς φίλου, ἤν που ἀκούσῃ,
ἠδ' ἵνα μιν κλέος ἐσθλὸν ἐν ἀνθρώποισιν ἔχῃσιν.»
\nstanza
ὣς εἰποῦσ' ὑπὸ ποσσὶν ἐδήσατο καλὰ πέδιλα,
ἀμβρόσια χρύσεια, τά μιν φέρον ἠμὲν ἐφ' ὑγρὴν
ἠδ' ἐπ' ἀπείρονα γαῖαν ἅμα πνοιῇσ' ἀνέμοιο.
εἵλετο δ' ἄλκιμον ἔγχος, ἀκαχμένον ὀξέϊ χαλκῷ,
βριθὺ μέγα στιβαρόν, τῷ δάμνησι στίχας ἀνδρῶν    \lineno{100}
ἡρώων, τοῖσίν τε κοτέσσεται ὀβριμοπάτρη,
βῆ δὲ κατ' Οὐλύμποιο καρήνων ἀΐξασα,
στῆ δ' Ἰθάκης ἐνὶ δήμῳ ἐπὶ προθύροισ' Ὀδυσῆος,
οὐδοῦ ἐπ' αὐλείου· παλάμῃ δ' ἔχε χάλκεον ἔγχος,
εἰδομένη ξείνῳ, Ταφίων ἡγήτορι, Μέντῃ.
εὗρε δ' ἄρα μνηστῆρας ἀγήνορας· οἱ μὲν ἔπειτα
πεσσοῖσι προπάροιθε θυράων θυμὸν ἔτερπον,
ἥμενοι ἐν ῥινοῖσι βοῶν, οὓς ἔκτανον αὐτοί.
κήρυκες δ' αὐτοῖσι καὶ ὀτρηροὶ θεράποντες
οἱ μὲν ἄρ' οἶνον ἔμισγον ἐνὶ κρητῆρσι καὶ ὕδωρ,    \lineno{110}
οἱ δ' αὖτε σπόγγοισι πολυτρήτοισι τραπέζας
νίζον καὶ πρότιθεν, τοὶ δὲ κρέα πολλὰ δατεῦντο.
\nstanza

τὴν δὲ πολὺ πρῶτος ἴδε Τηλέμαχος θεοειδής·
ἧστο γὰρ ἐν μνηστῆρσι φίλον τετιημένος ἦτορ,
ὀσσόμενος πατέρ' ἐσθλὸν ἐνὶ φρεσίν, εἴ ποθεν ἐλθὼν
μνηστήρων τῶν μὲν σκέδασιν κατὰ δώματα θείη,
τιμὴν δ' αὐτὸς ἔχοι καὶ κτήμασιν οἷσιν ἀνάσσοι.
τὰ φρονέων μνηστῆρσι μεθήμενος εἴσιδ' Ἀθήνην,
βῆ δ' ἰθὺς προθύροιο, νεμεσσήθη δ' ἐνὶ θυμῷ
ξεῖνον δηθὰ θύρῃσιν ἐφεστάμεν· ἐγγύθι δὲ στὰς    \lineno{120}
χεῖρ' ἕλε δεξιτερὴν καὶ ἐδέξατο χάλκεον ἔγχος,
καί μιν φωνήσας ἔπεα πτερόεντα προσηύδα·
\nstanza
«χαῖρε, ξεῖνε, παρ' ἄμμι φιλήσεαι· αὐτὰρ ἔπειτα
δείπνου πασσάμενος μυθήσεαι ὅττεό σε χρή.»
\nstanza
ὣς εἰπὼν ἡγεῖθ', ἡ δ' ἕσπετο Παλλὰς Ἀθήνη.
οἱ δ' ὅτε δή ῥ' ἔντοσθεν ἔσαν δόμου ὑψηλοῖο,
ἔγχος μέν ῥ' ἔστησε φέρων πρὸς κίονα μακρὴν
δουροδόκης ἔντοσθεν ἐϋξόου, ἔνθα περ ἄλλα
ἔγχε' Ὀδυσσῆος ταλασίφρονος ἵστατο πολλά,
αὐτὴν δ' ἐς θρόνον εἷσεν ἄγων, ὑπὸ λῖτα πετάσσας,    \lineno{130}
καλὸν δαιδάλεον· ὑπὸ δὲ θρῆνυς ποσὶν ἦεν.
πὰρ δ' αὐτὸς κλισμὸν θέτο ποικίλον, ἔκτοθεν ἄλλων
μνηστήρων, μὴ ξεῖνος ἀνιηθεὶς ὀρυμαγδῷ
δείπνῳ ἀηδήσειεν, ὑπερφιάλοισι μετελθών,
ἠδ' ἵνα μιν περὶ πατρὸς ἀποιχομένοιο ἔροιτο.
χέρνιβα δ' ἀμφίπολος προχόῳ ἐπέχευε φέρουσα
καλῇ χρυσείῃ, ὑπὲρ ἀργυρέοιο λέβητος,
νίψασθαι· παρὰ δὲ ξεστὴν ἐτάνυσσε τράπεζαν.
σῖτον δ' αἰδοίη ταμίη παρέθηκε φέρουσα,
εἴδατα πόλλ' ἐπιθεῖσα, χαριζομένη παρεόντων·    \lineno{140}
δαιτρὸς δὲ κρειῶν πίνακας παρέθηκεν ἀείρας
παντοίων, παρὰ δέ σφι τίθει χρύσεια κύπελλα,
κῆρυξ δ' αὐτοῖσιν θάμ' ἐπῴχετο οἰνοχοεύων.
\nstanza
ἐς δ' ἦλθον μνηστῆρες ἀγήνορες· οἱ μὲν ἔπειτα
ἑξείης ἕζοντο κατὰ κλισμούς τε θρόνους τε.
τοῖσι δὲ κήρυκες μὲν ὕδωρ ἐπὶ χεῖρας ἔχευαν,
σῖτον δὲ δμῳαὶ παρενήεον ἐν κανέοισι,
κοῦροι δὲ κρητῆρας ἐπεστέψαντο ποτοῖο.
οἱ δ' ἐπ' ὀνείαθ' ἑτοῖμα προκείμενα χεῖρας ἴαλλον.
αὐτὰρ ἐπεὶ πόσιος καὶ ἐδητύος ἐξ ἔρον ἕντο    \lineno{150}
μνηστῆρες, τοῖσιν μὲν ἐνὶ φρεσὶν ἄλλα μεμήλει,
μολπή τ' ὀρχηστύς τε· τὰ γάρ τ' ἀναθήματα δαιτός.
κῆρυξ δ' ἐν χερσὶν κίθαριν περικαλλέα θῆκε
Φημίῳ, ὅς ῥ' ἤειδε παρὰ μνηστῆρσιν ἀνάγκῃ.
ἦ τοι ὁ φορμίζων ἀνεβάλλετο καλὸν ἀείδειν,
αὐτὰρ Τηλέμαχος προσέφη γλαυκῶπιν Ἀθήνην,
ἄγχι σχὼν κεφαλήν, ἵνα μὴ πευθοίαθ' οἱ ἄλλοι·
\nstanza
«ξεῖνε φίλ', ἦ καί μοι νεμεσήσεαι ὅττι κεν εἴπω;
τούτοισιν μὲν ταῦτα μέλει, κίθαρις καὶ ἀοιδή,
ῥεῖ', ἐπεὶ ἀλλότριον βίοτον νήποινον ἔδουσιν,    \lineno{160}
ἀνέρος, οὗ δή που λεύκ' ὀστέα πύθεται ὄμβρῳ
κείμεν' ἐπ' ἠπείρου, ἢ εἰν ἁλὶ κῦμα κυλίνδει.
εἰ κεῖνόν γ' Ἰθάκηνδε ἰδοίατο νοστήσαντα,
πάντες κ' ἀρησαίατ' ἐλαφρότεροι πόδας εἶναι
ἢ ἀφνειότεροι χρυσοῖό τε ἐσθῆτός τε.
νῦν δ' ὁ μὲν ὣς ἀπόλωλε κακὸν μόρον, οὐδέ τις ἥμιν
θαλπωρή, εἴ πέρ τις ἐπιχθονίων ἀνθρώπων
φῇσιν ἐλεύσεσθαι· τοῦ δ' ὤλετο νόστιμον ἦμαρ.
ἀλλ' ἄγε μοι τόδε εἰπὲ καὶ ἀτρεκέως κατάλεξον·
τίς πόθεν εἰς ἀνδρῶν; πόθι τοι πόλις ἠδὲ τοκῆες;    \lineno{170}
ὁπποίης τ' ἐπὶ νηὸς ἀφίκεο; πῶς δέ σε ναῦται
ἤγαγον εἰς Ἰθάκην; τίνες ἔμμεναι εὐχετόωντο;
οὐ μὲν γάρ τί σε πεζὸν ὀΐομαι ἐνθάδ' ἱκέσθαι.
καί μοι τοῦτ' ἀγόρευσον ἐτήτυμον, ὄφρ' ἐῢ εἰδῶ,
ἠὲ νέον μεθέπεις, ἦ καὶ πατρώϊός ἐσσι
ξεῖνος, ἐπεὶ πολλοὶ ἴσαν ἀνέρες ἡμέτερον δῶ
ἄλλοι, ἐπεὶ καὶ κεῖνος ἐπίστροφος ἦν ἀνθρώπων.»
\nstanza
τὸν δ' αὖτε προσέειπε θεὰ γλαυκῶπις Ἀθήνη·
«τοιγὰρ ἐγώ τοι ταῦτα μάλ' ἀτρεκέως ἀγορεύσω.
Μέντης Ἀγχιάλοιο δαΐφρονος εὔχομαι εἶναι    \lineno{180}
υἱός, ἀτὰρ Ταφίοισι φιληρέτμοισιν ἀνάσσω.
νῦν δ' ὧδε ξὺν νηῒ κατήλυθον ἠδ' ἑτάροισι,
πλέων ἐπὶ οἴνοπα πόντον ἐπ' ἀλλοθρόους ἀνθρώπους,
ἐς Τεμέσην μετὰ χαλκόν, ἄγω δ' αἴθωνα σίδηρον.
νηῦς δέ μοι ἥδ' ἕστηκεν ἐπ' ἀγροῦ νόσφι πόληος,
ἐν λιμένι Ῥείθρῳ, ὑπὸ Νηΐῳ ὑλήεντι.
ξεῖνοι δ' ἀλλήλων πατρώϊοι εὐχόμεθ' εἶναι
ἐξ ἀρχῆς, εἴ πέρ τε γέροντ' εἴρηαι ἐπελθὼν
Λαέρτην ἥρωα, τὸν οὐκέτι φασὶ πόλινδε
ἔρχεσθ', ἀλλ' ἀπάνευθεν ἐπ' ἀγροῦ πήματα πάσχειν    \lineno{190}
γρηῒ σὺν ἀμφιπόλῳ, ἥ οἱ βρῶσίν τε πόσιν τε
παρτιθεῖ, εὖτ' ἄν μιν κάματος κατὰ γυῖα λάβῃσιν
ἑρπύζοντ' ἀνὰ γουνὸν ἀλῳῆς οἰνοπέδοιο.
νῦν δ' ἦλθον· δὴ γάρ μιν ἔφαντ' ἐπιδήμιον εἶναι,
σὸν πατέρ'· ἀλλά νυ τόν γε θεοὶ βλάπτουσι κελεύθου.
οὐ γάρ πω τέθνηκεν ἐπὶ χθονὶ δῖος Ὀδυσσεύς,
ἀλλ' ἔτι που ζωὸς κατερύκεται εὐρέϊ πόντῳ,
νήσῳ ἐν ἀμφιρύτῃ, χαλεποὶ δέ μιν ἄνδρες ἔχουσιν,
ἄγριοι, οἵ που κεῖνον ἐρυκανόωσ' ἀέκοντα.
αὐτὰρ νῦν τοι ἐγὼ μαντεύσομαι, ὡς ἐνὶ θυμῷ    \lineno{200}
ἀθάνατοι βάλλουσι καὶ ὡς τελέεσθαι ὀΐω,
οὔτε τι μάντις ἐὼν οὔτ' οἰωνῶν σάφα εἰδώς.
οὔ τοι ἔτι δηρόν γε φίλης ἀπὸ πατρίδος αἴης
ἔσσεται, οὐδ' εἴ πέρ τε σιδήρεα δέσματ' ἔχῃσι·
φράσσεται ὥς κε νέηται, ἐπεὶ πολυμήχανός ἐστιν.
ἀλλ' ἄγε μοι τόδε εἰπὲ καὶ ἀτρεκέως κατάλεξον,
εἰ δὴ ἐξ αὐτοῖο τόσος πάϊς εἰς Ὀδυσῆος.
αἰνῶς μὲν κεφαλήν τε καὶ ὄμματα καλὰ ἔοικας
κείνῳ, ἐπεὶ θαμὰ τοῖον ἐμισγόμεθ' ἀλλήλοισι,
πρίν γε τὸν ἐς Τροίην ἀναβήμεναι, ἔνθα περ ἄλλοι    \lineno{210}
Ἀργείων οἱ ἄριστοι ἔβαν κοίλῃσ' ἐνὶ νηυσίν·
ἐκ τοῦ δ' οὔτ' Ὀδυσῆα ἐγὼν ἴδον οὔτ' ἐμὲ κεῖνος.»
\nstanza
τὴν δ' αὖ Τηλέμαχος πεπνυμένος ἀντίον ηὔδα·
«τοιγὰρ ἐγώ τοι, ξεῖνε, μάλ' ἀτρεκέως ἀγορεύσω.
μήτηρ μέν τέ μέ φησι τοῦ ἔμμεναι, αὐτὰρ ἐγώ γε
οὐκ οἶδ'· οὐ γάρ πώ τις ἑὸν γόνον αὐτὸς ἀνέγνω.
ὡς δὴ ἐγώ γ' ὄφελον μάκαρός νύ τευ ἔμμεναι υἱὸς
ἀνέρος, ὃν κτεάτεσσιν ἑοῖσ' ἔπι γῆρας ἔτετμε.
νῦν δ' ὃς ἀποτμότατος γένετο θνητῶν ἀνθρώπων,
τοῦ μ' ἔκ φασι γενέσθαι, ἐπεὶ σύ με τοῦτ' ἐρεείνεις.»    \lineno{220}
\nstanza
τὸν δ' αὖτε προσέειπε θεὰ γλαυκῶπις Ἀθήνη·
«οὐ μέν τοι γενεήν γε θεοὶ νώνυμνον ὀπίσσω
θῆκαν, ἐπεὶ σέ γε τοῖον ἐγείνατο Πηνελόπεια.
ἀλλ' ἄγε μοι τόδε εἰπὲ καὶ ἀτρεκέως κατάλεξον·
τίς δαίς, τίς δὲ ὅμιλος ὅδ' ἔπλετο; τίπτε δέ σε χρεώ;
εἰλαπίνη ἦε γάμος; ἐπεὶ οὐκ ἔρανος τάδε γ' ἐστίν,
ὥς τέ μοι ὑβρίζοντες ὑπερφιάλως δοκέουσι
δαίνυσθαι κατὰ δῶμα. νεμεσσήσαιτό κεν ἀνὴρ
αἴσχεα πόλλ' ὁρόων, ὅς τις πινυτός γε μετέλθοι.»
\nstanza
τὴν δ' αὖ Τηλέμαχος πεπνυμένος ἀντίον ηὔδα·    \lineno{230}
«ξεῖν', ἐπεὶ ἂρ δὴ ταῦτά μ' ἀνείρεαι ἠδὲ μεταλλᾷς,
μέλλεν μέν ποτε οἶκος ὅδ' ἀφνειὸς καὶ ἀμύμων
ἔμμεναι, ὄφρ' ἔτι κεῖνος ἀνὴρ ἐπιδήμιος ἦεν·
νῦν δ' ἑτέρως ἐβόλοντο θεοὶ κακὰ μητιόωντες,
οἳ κεῖνον μὲν ἄϊστον ἐποίησαν περὶ πάντων
ἀνθρώπων, ἐπεὶ οὔ κε θανόντι περ ὧδ' ἀκαχοίμην,
εἰ μετὰ οἷσ' ἑτάροισι δάμη Τρώων ἐνὶ δήμῳ,
ἠὲ φίλων ἐν χερσίν, ἐπεὶ πόλεμον τολύπευσε.
τῶ κέν οἱ τύμβον μὲν ἐποίησαν Παναχαιοί,
ἠδέ κε καὶ ᾧ παιδὶ μέγα κλέος ἤρατ' ὀπίσσω.    \lineno{240}
νῦν δέ μιν ἀκλειῶς Ἅρπυιαι ἀνηρέψαντο·
οἴχετ' ἄϊστος ἄπυστος, ἐμοὶ δ' ὀδύνας τε γόους τε
κάλλιπεν· οὐδέ τι κεῖνον ὀδυρόμενος στεναχίζω
οἶον, ἐπεί νύ μοι ἄλλα θεοὶ κακὰ κήδε' ἔτευξαν.
ὅσσοι γὰρ νήσοισιν ἐπικρατέουσιν ἄριστοι,
Δουλιχίῳ τε Σάμῃ τε καὶ ὑλήεντι Ζακύνθῳ,
ἠδ' ὅσσοι κραναὴν Ἰθάκην κάτα κοιρανέουσι,
τόσσοι μητέρ' ἐμὴν μνῶνται, τρύχουσι δὲ οἶκον.
ἡ δ' οὔτ' ἀρνεῖται στυγερὸν γάμον οὔτε τελευτὴν
ποιῆσαι δύναται· τοὶ δὲ φθινύθουσιν ἔδοντες    \lineno{250}
οἶκον ἐμόν· τάχα δή με διαῤῥαίσουσι καὶ αὐτόν.»
\nstanza
τὸν δ' ἐπαλαστήσασα προσηύδα Παλλὰς Ἀθήνη·
«ὢ πόποι, ἦ δὴ πολλὸν ἀποιχομένου Ὀδυσῆος
δεύῃ, ὅ κε μνηστῆρσιν ἀναιδέσι χεῖρας ἐφείη.
εἰ γὰρ νῦν ἐλθὼν δόμου ἐν πρώτῃσι θύρῃσι
σταίη, ἔχων πήληκα καὶ ἀσπίδα καὶ δύο δοῦρε,
τοῖος ἐὼν οἷόν μιν ἐγὼ τὰ πρῶτ' ἐνόησα
οἴκῳ ἐν ἡμετέρῳ πίνοντά τε τερπόμενόν τε,
\nstanza
ἐξ Ἐφύρης ἀνιόντα παρ' Ἴλου Μερμερίδαο· -
ᾤχετο γὰρ καὶ κεῖσε θοῆς ἐπὶ νηὸς Ὀδυσσεὺς    \lineno{260}
φάρμακον ἀνδροφόνον διζήμενος, ὄφρα οἱ εἴη
ἰοὺς χρίεσθαι χαλκήρεας· ἀλλ' ὁ μὲν οὔ οἱ
δῶκεν, ἐπεί ῥα θεοὺς νεμεσίζετο αἰὲν ἐόντας,
ἀλλὰ πατήρ οἱ δῶκεν ἐμός· φιλέεσκε γὰρ αἰνῶς· -
τοῖος ἐὼν μνηστῆρσιν ὁμιλήσειεν Ὀδυσσεύς·
πάντες κ' ὠκύμοροί τε γενοίατο πικρόγαμοί τε.
ἀλλ' ἦ τοι μὲν ταῦτα θεῶν ἐν γούνασι κεῖται,
ἤ κεν νοστήσας ἀποτείσεται, ἦε καὶ οὐκί,
οἷσιν ἐνὶ μεγάροισι· σὲ δὲ φράζεσθαι ἄνωγα,
ὅππως κε μνηστῆρας ἀπώσεαι ἐκ μεγάροιο.    \lineno{270}
εἰ δ' ἄγε νῦν ξυνίει καὶ ἐμῶν ἐμπάζεο μύθων·
αὔριον εἰς ἀγορὴν καλέσας ἥρωας Ἀχαιοὺς
μῦθον πέφραδε πᾶσι, θεοὶ δ' ἐπὶ μάρτυροι ἔστων.
μνηστῆρας μὲν ἐπὶ σφέτερα σκίδνασθαι ἄνωχθι,
μητέρα δ', εἴ οἱ θυμὸς ἐφορμᾶται γαμέεσθαι,
ἂψ ἴτω ἐς μέγαρον πατρὸς μέγα δυναμένοιο·
οἱ δὲ γάμον τεύξουσι καὶ ἀρτυνέουσιν ἔεδνα
πολλὰ μάλ', ὅσσα ἔοικε φίλης ἐπὶ παιδὸς ἕπεσθαι.
σοὶ δ' αὐτῷ πυκινῶς ὑποθήσομαι, αἴ κε πίθηαι·
νῆ' ἄρσας ἐρέτῃσιν ἐείκοσιν, ἥ τις ἀρίστη,    \lineno{280}
ἔρχεο πευσόμενος πατρὸς δὴν οἰχομένοιο,
ἤν τίς τοι εἴπῃσι βροτῶν, ἢ ὄσσαν ἀκούσῃς
ἐκ Διός, ἥ τε μάλιστα φέρει κλέος ἀνθρώποισι.
πρῶτα μὲν ἐς Πύλον ἐλθὲ καὶ εἴρεο Νέστορα δῖον,
κεῖθεν δὲ Σπάρτηνδε παρὰ ξανθὸν Μενέλαον·
ὃς γὰρ δεύτατος ἦλθεν Ἀχαιῶν χαλκοχιτώνων.
εἰ μέν κεν πατρὸς βίοτον καὶ νόστον ἀκούσῃς,
ἦ τ' ἂν τρυχόμενός περ ἔτι τλαίης ἐνιαυτόν·
εἰ δέ κε τεθνηῶτος ἀκούσῃς μηδ' ἔτ' ἐόντος,
νοστήσας δὴ ἔπειτα φίλην ἐς πατρίδα γαῖαν    \lineno{290}
σῆμά τέ οἱ χεῦαι καὶ ἐπὶ κτέρεα κτερεΐξαι
πολλὰ μάλ', ὅσσα ἔοικε, καὶ ἀνέρι μητέρα δοῦναι.
αὐτὰρ ἐπὴν δὴ ταῦτα τελευτήσῃς τε καὶ ἕρξῃς,
φράζεσθαι δὴ ἔπειτα κατὰ φρένα καὶ κατὰ θυμόν,
ὅππως κε μνηστῆρας ἐνὶ μεγάροισι τεοῖσι
κτείνῃς ἠὲ δόλῳ ἢ ἀμφαδόν· οὐδέ τί σε χρὴ
νηπιάας ὀχέειν, ἐπεὶ οὐκέτι τηλίκος ἐσσί.
ἦ οὐκ ἀΐεις οἷον κλέος ἔλλαβε δῖος Ὀρέστης
πάντας ἐπ' ἀνθρώπους, ἐπεὶ ἔκτανε πατροφονῆα,
Αἴγισθον δολόμητιν, ὅ οἱ πατέρα κλυτὸν ἔκτα;    \lineno{300}
καὶ σύ, φίλος, μάλα γάρ σ' ὁρόω καλόν τε μέγαν τε,
ἄλκιμος ἔσσ', ἵνα τίς σε καὶ ὀψιγόνων ἐῢ εἴπῃ.
αὐτὰρ ἐγὼν ἐπὶ νῆα θοὴν κατελεύσομαι ἤδη
ἠδ' ἑτάρους, οἵ πού με μάλ' ἀσχαλόωσι μένοντες·
σοὶ δ' αὐτῷ μελέτω, καὶ ἐμῶν ἐμπάζεο μύθων.»
\nstanza
τὴν δ' αὖ Τηλέμαχος πεπνυμένος ἀντίον ηὔδα·
«ξεῖν', ἦ τοι μὲν ταῦτα φίλα φρονέων ἀγορεύεις,
ὥς τε πατὴρ ᾧ παιδί, καὶ οὔ ποτε λήσομαι αὐτῶν.
ἀλλ' ἄγε νῦν ἐπίμεινον, ἐπειγόμενός περ ὁδοῖο,
ὄφρα λοεσσάμενός τε τεταρπόμενός τε φίλον κῆρ    \lineno{310}
δῶρον ἔχων ἐπὶ νῆα κίῃς, χαίρων ἐνὶ θυμῷ,
τιμῆεν, μάλα καλόν, ὅ τοι κειμήλιον ἔσται
ἐξ ἐμεῦ, οἷα φίλοι ξεῖνοι ξείνοισι διδοῦσι.»
\nstanza
τὸν δ' ἠμείβετ' ἔπειτα θεὰ γλαυκῶπις Ἀθήνη·
«μή μ' ἔτι νῦν κατέρυκε, λιλαιόμενόν περ ὁδοῖο·
δῶρον δ' ὅττι κέ μοι δοῦναι φίλον ἦτορ ἀνώγῃ,
αὖτις ἀνερχομένῳ δόμεναι οἶκόνδε φέρεσθαι,
καὶ μάλα καλὸν ἑλών· σοὶ δ' ἄξιον ἔσται ἀμοιβῆς.»
\nstanza
ἡ μὲν ἄρ' ὣς εἰποῦσ' ἀπέβη γλαυκῶπις Ἀθήνη,
ὄρνις δ' ὣς ἀνόπαια διέπτατο· τῷ δ' ἐνὶ θυμῷ    \lineno{320}
θῆκε μένος καὶ θάρσος, ὑπέμνησέν τέ ἑ πατρὸς
μᾶλλον ἔτ' ἢ τὸ πάροιθεν. ὁ δὲ φρεσὶν ᾗσι νοήσας
θάμβησεν κατὰ θυμόν· ὀΐσατο γὰρ θεὸν εἶναι.
αὐτίκα δὲ μνηστῆρας ἐπῴχετο ἰσόθεος φώς.
\nstanza
τοῖσι δ' ἀοιδὸς ἄειδε περικλυτός, οἱ δὲ σιωπῇ
εἵατ' ἀκούοντες· ὁ δ' Ἀχαιῶν νόστον ἄειδε
λυγρόν, ὃν ἐκ Τροίης ἐπετείλατο Παλλὰς Ἀθήνη.
\nstanza
τοῦ δ' ὑπερωϊόθεν φρεσὶ σύνθετο θέσπιν ἀοιδὴν
κούρη Ἰκαρίοιο, περίφρων Πηνελόπεια·
κλίμακα δ' ὑψηλὴν κατεβήσετο οἷο δόμοιο,    \lineno{330}
οὐκ οἴη, ἅμα τῇ γε καὶ ἀμφίπολοι δύ' ἕποντο.
ἡ δ' ὅτε δὴ μνηστῆρας ἀφίκετο δῖα γυναικῶν,
στῆ ῥα παρὰ σταθμὸν τέγεος πύκα ποιητοῖο,
ἄντα παρειάων σχομένη λιπαρὰ κρήδεμνα·
ἀμφίπολος δ' ἄρα οἱ κεδνὴ ἑκάτερθε παρέστη.
δακρύσασα δ' ἔπειτα προσηύδα θεῖον ἀοιδόν·
\nstanza
«Φήμιε, πολλὰ γὰρ ἄλλα βροτῶν θελκτήρια οἶδας
ἔργ' ἀνδρῶν τε θεῶν τε, τά τε κλείουσιν ἀοιδοί·
τῶν ἕν γέ σφιν ἄειδε παρήμενος, οἱ δὲ σιωπῇ
οἶνον πινόντων· ταύτης δ' ἀποπαύε' ἀοιδῆς    \lineno{340}
λυγρῆς, ἥ τέ μοι αἰὲν ἐνὶ στήθεσσι φίλον κῆρ
τείρει, ἐπεί με μάλιστα καθίκετο πένθος ἄλαστον.
τοίην γὰρ κεφαλὴν ποθέω μεμνημένη αἰεὶ
ἀνδρός, τοῦ κλέος εὐρὺ καθ' Ἑλλάδα καὶ μέσον Ἄργος.»
\nstanza
τὴν δ' αὖ Τηλέμαχος πεπνυμένος ἀντίον ηὔδα·
«μῆτερ ἐμή, τί τ' ἄρα φθονέεις ἐρίηρον ἀοιδὸν
τέρπειν ὅππῃ οἱ νόος ὄρνυται; οὔ νύ τ' ἀοιδοὶ
αἴτιοι, ἀλλά ποθι Ζεὺς αἴτιος, ὅς τε δίδωσιν
ἀνδράσιν ἀλφηστῇσιν ὅπως ἐθέλῃσιν ἑκάστῳ.
τούτῳ δ' οὐ νέμεσις Δαναῶν κακὸν οἶτον ἀείδειν·    \lineno{350}
τὴν γὰρ ἀοιδὴν μᾶλλον ἐπικλείουσ' ἄνθρωποι,
ἥ τις ἀϊόντεσσι νεωτάτη ἀμφιπέληται.
σοὶ δ' ἐπιτολμάτω κραδίη καὶ θυμὸς ἀκούειν·
οὐ γὰρ Ὀδυσσεὺς οἶος ἀπώλεσε νόστιμον ἦμαρ
ἐν Τροίῃ, πολλοὶ δὲ καὶ ἄλλοι φῶτες ὄλοντο.
ἀλλ' εἰς οἶκον ἰοῦσα τὰ σ' αὐτῆς ἔργα κόμιζε,
ἱστόν τ' ἠλακάτην τε, καὶ ἀμφιπόλοισι κέλευε
ἔργον ἐποίχεσθαι· μῦθος δ' ἄνδρεσσι μελήσει
πᾶσι, μάλιστα δ' ἐμοί· τοῦ γὰρ κράτος ἔστ' ἐνὶ οἴκῳ.»
\nstanza
ἡ μὲν θαμβήσασα πάλιν οἶκόνδε βεβήκει·    \lineno{360}
παιδὸς γὰρ μῦθον πεπνυμένον ἔνθετο θυμῷ.
ἐς δ' ὑπερῷ' ἀναβᾶσα σὺν ἀμφιπόλοισι γυναιξὶ
κλαῖεν ἔπειτ' Ὀδυσῆα, φίλον πόσιν, ὄφρα οἱ ὕπνον
ἡδὺν ἐπὶ βλεφάροισι βάλε γλαυκῶπις Ἀθήνη.
\nstanza
μνηστῆρες δ' ὁμάδησαν ἀνὰ μέγαρα σκιόεντα·
πάντες δ' ἠρήσαντο παραὶ λεχέεσσι κλιθῆναι.
τοῖσι δὲ Τηλέμαχος πεπνυμένος ἤρχετο μύθων·
\nstanza
«μητρὸς ἐμῆς μνηστῆρες, ὑπέρβιον ὕβριν ἔχοντες,
νῦν μὲν δαινύμενοι τερπώμεθα, μηδὲ βοητὺς
ἔστω, ἐπεὶ τό γε καλὸν ἀκουέμεν ἐστὶν ἀοιδοῦ    \lineno{370}
τοιοῦδ' οἷος ὅδ' ἐστί, θεοῖσ' ἐναλίγκιος αὐδήν.
ἠῶθεν δ' ἀγορήνδε καθεζώμεσθα κιόντες
πάντες, ἵν' ὕμιν μῦθον ἀπηλεγέως ἀποείπω,
ἐξιέναι μεγάρων· ἄλλας δ' ἀλεγύνετε δαῖτας,
ὑμὰ κτήματ' ἔδοντες, ἀμειβόμενοι κατὰ οἴκους.
εἰ δ' ὕμιν δοκέει τόδε λωΐτερον καὶ ἄμεινον
ἔμμεναι, ἀνδρὸς ἑνὸς βίοτον νήποινον ὀλέσθαι,
κείρετ'· ἐγὼ δὲ θεοὺς ἐπιβώσομαι αἰὲν ἐόντας,
αἴ κέ ποθι Ζεὺς δῷσι παλίντιτα ἔργα γενέσθαι·
νήποινοί κεν ἔπειτα δόμων ἔντοσθεν ὄλοισθε.»    \lineno{380}
\nstanza
ὣς ἔφαθ', οἱ δ' ἄρα πάντες ὀδὰξ ἐν χείλεσι φύντες
Τηλέμαχον θαύμαζον, ὃ θαρσαλέως ἀγόρευε.
\nstanza
τὸν δ' αὖτ' Ἀντίνοος προσέφη, Εὐπείθεος υἱός·
«Τηλέμαχ', ἦ μάλα δή σε διδάσκουσιν θεοὶ αὐτοὶ
ὑψαγόρην τ' ἔμεναι καὶ θαρσαλέως ἀγορεύειν.
μὴ σέ γ' ἐν ἀμφιάλῳ Ἰθάκῃ βασιλῆα Κρονίων
ποιήσειεν, ὅ τοι γενεῇ πατρώϊόν ἐστιν.»
\nstanza
τὸν δ' αὖ Τηλέμαχος πεπνυμένος ἀντίον ηὔδα·
«Ἀντίνο', εἴ πέρ μοι καὶ ἀγάσσεαι ὅττι κεν εἴπω,
καί κεν τοῦτ' ἐθέλοιμι Διός γε διδόντος ἀρέσθαι.    \lineno{390}
ἦ φῂς τοῦτο κάκιστον ἐν ἀνθρώποισι τετύχθαι;
οὐ μὲν γάρ τι κακὸν βασιλευέμεν· αἶψά τέ οἱ δῶ
ἀφνειὸν πέλεται καὶ τιμηέστερος αὐτός.
ἀλλ' ἦ τοι βασιλῆες Ἀχαιῶν εἰσὶ καὶ ἄλλοι
πολλοὶ ἐν ἀμφιάλῳ Ἰθάκῃ, νέοι ἠδὲ παλαιοί,
τῶν κέν τις τόδ' ἔχῃσιν, ἐπεὶ θάνε δῖος Ὀδυσσεύς·
αὐτὰρ ἐγὼν οἴκοιο ἄναξ ἔσομ' ἡμετέροιο
καὶ δμώων, οὕς μοι ληΐσσατο δῖος Ὀδυσσεύς.»
\nstanza
τὸν δ' αὖτ' Εὐρύμαχος, Πολύβου πάϊς, ἀντίον ηὔδα·
«Τηλέμαχ', ἦ τοι ταῦτα θεῶν ἐν γούνασι κεῖται,    \lineno{400}
ὅς τις ἐν ἀμφιάλῳ Ἰθάκῃ βασιλεύσει Ἀχαιῶν·
κτήματα δ' αὐτὸς ἔχοις καὶ δώμασι σοῖσιν ἀνάσσοις.
μὴ γὰρ ὅ γ' ἔλθοι ἀνήρ, ὅς τίς σ' ἀέκοντα βίηφι
κτήματ' ἀποῤῥαίσει', Ἰθάκης ἔτι ναιεταούσης.
\nstanza
ἀλλ' ἐθέλω σε, φέριστε, περὶ ξείνοιο ἐρέσθαι,
ὁππόθεν οὗτος ἀνήρ· ποίης δ' ἐξ εὔχεται εἶναι
γαίης; ποῦ δέ νύ οἱ γενεὴ καὶ πατρὶς ἄρουρα;
ἠέ τιν' ἀγγελίην πατρὸς φέρει ἐρχομένοιο,
ἦ ἑὸν αὐτοῦ χρεῖος ἐελδόμενος τόδ' ἱκάνει;
οἷον ἀναΐξας ἄφαρ οἴχεται, οὐδ' ὑπέμεινε    \lineno{410}
γνώμεναι· οὐ μὲν γάρ τι κακῷ εἰς ὦπα ἐῴκει.»
\nstanza
τὸν δ' αὖ Τηλέμαχος πεπνυμένος ἀντίον ηὔδα·
«Εὐρύμαχ', ἦ τοι νόστος ἀπώλετο πατρὸς ἐμοῖο·
οὔτ' οὖν ἀγγελίῃ ἔτι πείθομαι, εἴ ποθεν ἔλθοι,
οὔτε θεοπροπίης ἐμπάζομαι, ἥν τινα μήτηρ
ἐς μέγαρον καλέσασα θεοπρόπον ἐξερέηται.
ξεῖνος δ' οὗτος ἐμὸς πατρώϊος ἐκ Τάφου ἐστί,
Μέντης δ' Ἀγχιάλοιο δαΐφρονος εὔχεται εἶναι
υἱός, ἀτὰρ Ταφίοισι φιληρέτμοισιν ἀνάσσει.»
\nstanza
ὣς φάτο Τηλέμαχος, φρεσὶ δ' ἀθανάτην θεὸν ἔγνω.    \lineno{420}
οἱ δ' εἰς ὀρχηστύν τε καὶ ἱμερόεσσαν ἀοιδὴν
τρεψάμενοι τέρποντο, μένον δ' ἐπὶ ἕσπερον ἐλθεῖν.
τοῖσι δὲ τερπομένοισι μέλας ἐπὶ ἕσπερος ἦλθε·
δὴ τότε κακκείοντες ἔβαν οἶκόνδε ἕκαστος.
Τηλέμαχος δ', ὅθι οἱ θάλαμος περικαλλέος αὐλῆς
ὑψηλὸς δέδμητο, περισκέπτῳ ἐνὶ χώρῳ,
ἔνθ' ἔβη εἰς εὐνὴν πολλὰ φρεσὶ μερμηρίζων.
τῷ δ' ἄρ' ἅμ' αἰθομένας δαΐδας φέρε κεδνὰ ἰδυῖα
Εὐρύκλει', Ὦπος θυγάτηρ Πεισηνορίδαο,
τήν ποτε Λαέρτης πρίατο κτεάτεσσιν ἑοῖσι,    \lineno{430}
πρωθήβην ἔτ' ἐοῦσαν, ἐεικοσάβοια δ' ἔδωκεν,
ἶσα δέ μιν κεδνῇ ἀλόχῳ τίεν ἐν μεγάροισιν,
εὐνῇ δ' οὔ ποτ' ἔμικτο, χόλον δ' ἀλέεινε γυναικός·
ἥ οἱ ἅμ' αἰθομένας δαΐδας φέρε καί ἑ μάλιστα
δμῳάων φιλέεσκε καὶ ἔτρεφε τυτθὸν ἐόντα.
ὤϊξεν δὲ θύρας θαλάμου πύκα ποιητοῖο,
ἕζετο δ' ἐν λέκτρῳ, μαλακὸν δ' ἔκδυνε χιτῶνα·
καὶ τὸν μὲν γραίης πυκιμηδέος ἔμβαλε χερσίν.
ἡ μὲν τὸν πτύξασα καὶ ἀσκήσασα χιτῶνα,
πασσάλῳ ἀγκρεμάσασα παρὰ τρητοῖσι λέχεσσι,    \lineno{440}
βῆ ῥ' ἴμεν ἐκ θαλάμοιο, θύρην δ' ἐπέρυσσε κορώνῃ
ἀργυρέῃ, ἐπὶ δὲ κληῗδ' ἐτάνυσσεν ἱμάντι.
ἔνθ' ὅ γε παννύχιος, κεκαλυμμένος οἰὸς ἀώτῳ,
βούλευε φρεσὶν ᾗσιν ὁδόν, τὴν πέφραδ' Ἀθήνη.
\egroup
\bye